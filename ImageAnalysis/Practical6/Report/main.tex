\documentclass{article}
% Language setting
% Replace `english' with e.g. `spanish' to change the document language
\usepackage[english]{babel}

\usepackage{caption}
% Set page size and margins
% Replace `letterpaper' with `a4paper' for UK/EU standard size
\usepackage[letterpaper,top=2cm,bottom=2cm,left=3cm,right=3cm,marginparwidth=1.75cm,margin=1in]{geometry}
\usepackage{listings}

% Useful packages
\usepackage{amsmath}
\usepackage{graphicx}
\usepackage{subfig}
\usepackage{algorithm}
\usepackage{algpseudocode}
\usepackage{booktabs}
\usepackage{longtable}
\usepackage{caption}
\usepackage{array}
\setlength{\heavyrulewidth}{1.5pt}
\setlength{\abovetopsep}{4pt}


\usepackage[colorlinks=true, allcolors=blue]{hyperref}

\title{3D Shape, Pre-processing and Visualization}
\author{Megan Mirnalini Sundaram R, Sherry Usman}

\begin{document}
\maketitle

\section*{Question 1}
\subsection*{Slices and Aspect Ratio}
The image has 16 slices. The aspect ratio can be defined as the ratio of total width (total number of columns) and the total height (total number of rows). In that case it is 160/140 which is 1.143. 

\subsection*{Function to display the contents of the 3D image} \label{sec:contents-3D_image}
The code snippet below shows the function for displaying the slices of the image.  
\begin{lstlisting}
import numpy as np
import matplotlib.pyplot as plt 

num_slices = 16
fig, axes = plt.subplots(4,4, figsize =(12,12))
for i in range(num_slices):
  ax = axes[i // 4, i % 4]
  tif.seek(i)
  slice = np.array(tif)

  ax.imshow(slice)
  ax.axis('off')
  ax.set_title(f'Slice {i+1}')

plt.tight_layout()
plt.show()
\end{lstlisting}
The results of this function are show in Appendix: Figure \ref{fig:3d-plane-image}. 
\clearpage
\section*{Question 2}
\subsection*{Thresholding and best value}
Thresholding was done on the chromo.tiff image using the \textit{diplib} library in Python and using VAA3D software.\\Figure \ref{fig:thresholding} shows the result of the Otsu Threshold (\ref{fig:otsu}) and the result of Otsu threshold with median filter (\ref{fig:median_otsu}). 

\begin{figure}[h!]
\centering
\subfloat{\label{fig:otsu}\includegraphics[width=0.5\textwidth]{Report/Images/otsu_threshold_image.png}}
\subfloat{\label{fig:median_otsu}\includegraphics[width=0.5\textwidth]{Report/Images/median_filter.png}}
\caption{(a) Chromo.tiff with Otsu Threshold, and (b) Chromo.tiff with Median Filter and Otsu Threshold}
\label{fig:thresholding}
\end{figure}


Figure \ref{fig:vaa3dthresh} shows the result of the Vaa3D software. As seen in figure \ref{fig:vaa3dthresh}, the best threshold value is 55. 

\begin{figure}[h!]
    \centering
    \includegraphics[width=1\linewidth]{Report/Images/vaa3d_thresh.png}
    \caption{Chromo.tiff image thresholded in VAA3D}
    \label{fig:vaa3dthresh}
\end{figure}



\subsection*{Algorithm for Depth Cueing}
The figure \ref{fig:depth-cuing} shows the algorithm for depth cueing. As shown in the algorithm, 16 evenly spaced values are taken from the range 0 to 255 and the slices are thresholded with the values. Slices lower in the plane are thresholded with lower values and slices higher in the plane are thresholded iwth higher values. The slices are combined to create a 3-dimensional layered image. 
\begin{figure}[h!]
    \centering
    \includegraphics[width=0.8\linewidth]{Report/Images/depth_cuing.png}
    \caption{Algorithm for depth cueing }
    \label{fig:depth-cuing}
\end{figure}

\subsection*{Depth Cueing}
Figure \ref{fig:depth-cuing-results} shows the result of depth cueing. 
\begin{figure}[h!]
    \centering
    \includegraphics[width=1\linewidth]{Report/Images/mip.png}
    \caption{Chromo.tiff with depth cueing}
    \label{fig:depth-cuing-results}
\end{figure}

The first image is a raw image obtained before processing techniques such as thresholding, median filter and contrast stretching. Thus it looks out of focus, blurry and underexposed. The second image is the product of these image analysis methods and the original image. 

\begin{figure}[h!]
    \centering
    \includegraphics[width=0.5\linewidth]{Report/Images/max_intensity.png}
    \caption{The stacked result of MIP}
    \label{fig:mip}
\end{figure}

\clearpage
\section*{Question 3}
\subsection*{Maximum Projection and Alpha Function}
As instructed in the file, the contrast was varied. In addition to this, the thickness \textbf{Z-thick} was also varied, to understand the effect on the contrast. \\The video files were generated for both maximum projection and alpha function, and are submitted along with this file. 

\medskip

\textbf{Maximum Projection} : File Name - \textit{chromo-MIP\_Contrast.mp4}

\textbf{Alpha Projection} : File Name - \textit{chromo-Alpha\_Contrast.mp4}
\subsection*{Surface Visualization}
For the given image, the lower and upper ranges were varied between [0,75], [75,150] and [150,255]. In addition to varying the ranges, the mesh density was also varied between [25,50,75,100].
\begin{itemize}
    \item Figures \ref{fig:surface_vis_0_75-25}, \ref{fig:surface_vis_0_75-50}, \ref{fig:surface_vis_0_75-75} and \ref{fig:surface_vis_0_75-100} capture the results, when the lower range is 0 and the upper range is 75, with the mesh density as 25, 50, 75, 100 respectively. 
    \item Figures \ref{fig:surface_vis_75_150-25}, \ref{fig:surface_vis_75_150-50}, \ref{fig:surface_vis_75_150-75} and \ref{fig:surface_vis_75_150-100} capture the results, when the lower range is 75 and the upper range is 150, with the mesh density as 25, 50, 75, 100 respectively.
    \item Figures \ref{fig:surface_vis_150_255-25}, \ref{fig:surface_vis_150_255-50}, \ref{fig:surface_vis_150_255-75} and \ref{fig:surface_vis_150_255-100} capture the results, when the lower range is 150 and the upper range is 255, with the mesh density as 25, 50, 75, 100 respectively. 
\end{itemize}

\begin{figure}[h!]
    \centering
    \includegraphics[width=0.3\linewidth]{Report/Images/6.3.2/0-75,25.png}
    \includegraphics[width=0.3\linewidth]{Report/Images/6.3.2/0-75,25-rotated.png}
    \includegraphics[width=0.3\linewidth]{Report/Images/6.3.2/0-75,25-sliced.png}
    \caption{Threshold: 0-75, mesh density:25, original (left), rotated (middle), cross-section(right)}
    \label{fig:surface_vis_0_75-25}
\end{figure}

\begin{figure}[h!]
    \centering
    \includegraphics[width=0.3\linewidth]{Report/Images/6.3.2/0-75,50.png}
    \includegraphics[width=0.3\linewidth]{Report/Images/6.3.2/0-75,50-rotated.png}
    \includegraphics[width=0.3\linewidth]{Report/Images/6.3.2/0-75,50-sliced.png}
    \caption{Threshold: 0-75, mesh density:50, original (left), rotated (middle), cross-section(right)}
    \label{fig:surface_vis_0_75-50}
\end{figure}


\begin{figure}[h!]
    \centering
    \includegraphics[width=0.3\linewidth]{Report/Images/6.3.2/0-75,75.png}
    \includegraphics[width=0.3\linewidth]{Report/Images/6.3.2/0-75,75-rotated.png}
    \includegraphics[width=0.3\linewidth]{Report/Images/6.3.2/0-75,75-sliced.png}
    \caption{Threshold: 0-75, mesh density:75, original (left), rotated (middle), cross-section(right)}
    \label{fig:surface_vis_0_75-75}
\end{figure}

\begin{figure}[h!]
    \centering
    \includegraphics[width=0.3\linewidth]{Report/Images/6.3.2/0-75,100.png}
    \includegraphics[width=0.3\linewidth]{Report/Images/6.3.2/0-75,100-rotated.png}
    \includegraphics[width=0.3\linewidth]{Report/Images/6.3.2/0-75,100-sliced.png}
    \caption{Threshold: 0-75, mesh density:100, original (left), rotated (middle), cross-section(right)}
    \label{fig:surface_vis_0_75-100}
\end{figure}


\begin{figure}[h!]
    \centering
    \includegraphics[width=0.3\linewidth]{Report/Images/6.3.2/75-150,25.png}
    \includegraphics[width=0.3\linewidth]{Report/Images/6.3.2/75-150,25-rotated.png}
    \includegraphics[width=0.3\linewidth]{Report/Images/6.3.2/75-150,25-sliced.png}
    \caption{Threshold: 75-150, mesh density:25, original (left), rotated (middle), cross-section(right)}
    \label{fig:surface_vis_75_150-25}
\end{figure}


\begin{figure}[h!]
    \centering
    \includegraphics[width=0.3\linewidth]{Report/Images/6.3.2/75-150,50.png}
    \includegraphics[width=0.3\linewidth]{Report/Images/6.3.2/75-150,50-rotated.png}
    \includegraphics[width=0.3\linewidth]{Report/Images/6.3.2/75-150,50-sliced.png}
    \caption{Threshold: 75-150, mesh density:50, original (left), rotated (middle), cross-section(right)}
    \label{fig:surface_vis_75_150-50}
\end{figure}


\begin{figure}[h!]
    \centering
    \includegraphics[width=0.3\linewidth]{Report/Images/6.3.2/75-150,75.png}
    \includegraphics[width=0.3\linewidth]{Report/Images/6.3.2/75-150,75-rotated.png}
    \includegraphics[width=0.3\linewidth]{Report/Images/6.3.2/75-150,75-sliced.png}
    \caption{Threshold: 75-150, mesh density:75, original (left), rotated (middle), cross-section(right)}
    \label{fig:surface_vis_75_150-75}
\end{figure}


\begin{figure}[h!]
    \centering
    \includegraphics[width=0.3\linewidth]{Report/Images/6.3.2/75-150,100.png}
    \includegraphics[width=0.3\linewidth]{Report/Images/6.3.2/75-150,100-rotated.png}
    \includegraphics[width=0.3\linewidth]{Report/Images/6.3.2/75-150,100-sliced.png}
    \caption{Threshold: 75-150, mesh density:100, original (left), rotated (middle), cross-section(right)}
    \label{fig:surface_vis_75_150-100}
\end{figure}


\begin{figure}[h!]
    \centering
    \includegraphics[width=0.3\linewidth]{Report/Images/6.3.2/150-255,25.png}
    \includegraphics[width=0.3\linewidth]{Report/Images/6.3.2/150-255,25-rotated.png}
    \includegraphics[width=0.3\linewidth]{Report/Images/6.3.2/150-255,25-sliced.png}
    \caption{Threshold: 150-255, mesh density:25, original (left), rotated (middle), cross-section(right)}
    \label{fig:surface_vis_150_255-25}
\end{figure}

\begin{figure}[h!]
    \centering
    \includegraphics[width=0.3\linewidth]{Report/Images/6.3.2/150-255,50.png}
    \includegraphics[width=0.3\linewidth]{Report/Images/6.3.2/150-255,50-rotated.png}
    \includegraphics[width=0.3\linewidth]{Report/Images/6.3.2/150-255,50-sliced.png}
    \caption{Threshold: 150-255, mesh density:50, original (left), rotated (middle), cross-section(right)}
    \label{fig:surface_vis_150_255-50}
\end{figure}

\begin{figure}[h!]
    \centering
    \includegraphics[width=0.3\linewidth]{Report/Images/6.3.2/150-255,75.png}
    \includegraphics[width=0.3\linewidth]{Report/Images/6.3.2/150-255,75-rotated.png}
    \includegraphics[width=0.3\linewidth]{Report/Images/6.3.2/150-255,75-sliced.png}
    \caption{Threshold: 150-255, mesh density:75, original (left), rotated (middle), cross-section(right)}
    \label{fig:surface_vis_150_255-75}
\end{figure}


\begin{figure}[h!]
    \centering
    \includegraphics[width=0.3\linewidth]{Report/Images/6.3.2/150-255,100.png}
    \includegraphics[width=0.3\linewidth]{Report/Images/6.3.2/150-255,100-rotated.png}
    \includegraphics[width=0.3\linewidth]{Report/Images/6.3.2/150-255,100-sliced.png}
    \caption{Threshold: 75-150, mesh density:100, original (left), rotated (middle), cross-section(right)}
    \label{fig:surface_vis_150_255-100}
\end{figure}
\clearpage

\subsection*{Volume and Surface Visualization}
For this part, figures \ref{fig:median_otsu} and \ref{fig:mip} were used for surface and volume visualization.\\
Figure \ref{fig:depth-cue-volume-viz} shows the results of the volume visualization of the depth-cued image (\ref{fig:mip}). The threshold was set to 54, and the contrast was set to -23. This allowed to view the image properly, without any loss. Figure \ref{fig:depth-cue-surface-viz} shows the surface visualization for the same. \\
Figure \ref{fig:binary-image-volume-viz} shows the results of the binary image i.e., figure (\ref{fig:median_otsu}). The threshold was set to 45, and the contrast was set to +55. Since this is a binary image, the contrast was much higher. Figure \ref{fig:binary-image-surface-viz} shows the surface visualization for the same. \\


\begin{figure}[h!]
\centering
\subfloat{\label{fig:depth-cuing-volume_v1}\includegraphics[width=0.45\textwidth]{Report/Images/6.3-7/Depth_Cuing_VolumeViz_t54_c-23_view1.png}}
\vspace{5 mm}
\subfloat{\label{fig:depth-cuing-volume_v2}\includegraphics[width=0.45\textwidth]{Report/Images/6.3-7/Depth_Cuing_VolumeViz_t54_c-23_view2.png}}
\caption{Volume Visualization of the Depth-cued Image}
\label{fig:depth-cue-volume-viz}
\end{figure}
\begin{figure}[h!]
\centering
\subfloat{\label{fig:depth-cuing-surface_v1}\includegraphics[width=0.45\textwidth]{Report/Images/6.3-7/DepthCuing_SurfaceVisualization_view1.png}}
\vspace{5 mm}
\subfloat{\label{fig:depth-cuing-surface_v2}\includegraphics[width=0.45\textwidth]{Report/Images/6.3-7/DepthCuing_SurfaceVisualization_view2.png}}
\caption{Surface Visualization of the Depth-cued Image}
\label{fig:depth-cue-surface-viz}
\end{figure}

\begin{figure}[h!]
\centering
\subfloat{\label{fig:median-otsu-volume_v1}\includegraphics[width=0.45\textwidth]{Report/Images/6.3-7/BinaryImage_VolumeVisualization_t43_c55_view1.png}}
\vspace{5 mm}
\subfloat{\label{fig:median-otsu-volume_v2}\includegraphics[width=0.45\textwidth]{Report/Images/6.3-7/BinaryImage_VolumeVisualization_t43_c55_view2.png}}
\caption{Volume Visualization of the Binary Image}
\label{fig:binary-image-volume-viz}
\end{figure}

\begin{figure}[h!]
\centering
\subfloat{\label{fig:median-otsu-surface_v1}\includegraphics[width=0.45\textwidth]{Report/Images/6.3-7/BinaryImage_SurfaceVisualization_view1.png}}
\vspace{5 mm}
\subfloat{\label{fig:median-otsu-surface_v2}\includegraphics[width=0.45\textwidth]{Report/Images/6.3-7/BinaryImage_SurfaceVisualization_view2.png}}
\caption{Surface Visualization of the Binary Image}
\label{fig:binary-image-surface-viz}
\end{figure}

\clearpage
\section*{Question 4}
The acquired 3D images of zebrafish was first converted from  \textit{lif} to \textit{tiff} using the provided code snippet. It was loaded into Huygens and the microscopic parameters were set for all 4 images. 
The parameters were modified based on the notes taken during the lab session. 
\begin{figure}[h!]
    \centering
    \includegraphics[width=0.85\linewidth]{Report/Images/parameters_set.png}
    \caption{Microscopic Parameters}
    \label{fig:microscope-parameters}
\end{figure}
\subsection*{Static 3D Visualization}
The 3D visualization of the images was done using Huygens SFP Volume Renderer. The parameters were minimally modified, so as to show the untouched image. \\
Figures \ref{fig:zebrafish_0} , \ref{fig:zebrafish_1} were generated in 10x, and figure \ref{fig:zebrafish_2} was taken in 100x. Figure \ref{fig:zebrafish_tail} is an image of the tail of the zebrafish. 
\begin{figure}[h!]
    \centering
    \includegraphics[width=0.75\linewidth]{Report/Images/6.4-9/Image_0.png}
    \caption{3D Visualization of Zebrafish}
    \label{fig:zebrafish_0}
\end{figure}
\begin{figure}[h!]
    \centering
    \includegraphics[width=0.75\linewidth]{Report/Images/6.4-9/image_1.png}
    \caption{3D Visualization of Zebrafish}
    \label{fig:zebrafish_1}
\end{figure}
\begin{figure}[h!]
    \centering
    \includegraphics[width=0.75\linewidth]{Report/Images/6.4-9/image_2.png}
    \caption{3D Visualization of Zebrafish}
    \label{fig:zebrafish_2}
\end{figure}
\begin{figure}[h!]
    \centering
    \includegraphics[width=0.75\linewidth]{Report/Images/6.4-9/image_3.png}
    \caption{3D Visualization of Zebrafish tail}
    \label{fig:zebrafish_tail}
\end{figure}
\clearpage
\subsection*{Improving Signal Strength} \label{Signal_Strength}
For improving the signal strength, the channels of the image are separated, and the parameters are altered for every channel. \\
This is done by using SPF Volume Renderer in Huygens. 
\begin{table}[h!]
    \centering
    \begin{tabular}{|p{0.3\linewidth} | p{0.3\linewidth}|}
    \hline
       Surface Threshold & 30 \\   \hline
       Surface Seed & 100 \\  \hline
       Brightness & 50 \\  \hline
       Saturation & 100 \\ \hline
    \end{tabular}
    \label{tab:signal}
\end{table}
\subsubsection*{Image 1}
Apart from the parameters mentioned above, the values of the signal garbage volume was changed to 825508. 
\begin{figure}[h!]
\centering
\subfloat{\label{fig:Before Tuning 1}\includegraphics[width=0.45\textwidth]{Report/Images/original/Image_for_Huygens_0_rendering.jpg}}
\vspace{5 mm}
\subfloat{\label{fig:After Tuning 1}\includegraphics[width=0.45\textwidth]{Report/Images/cleaned-up/Image_for_Huygens_0_rendering.jpg}}
\caption{Processing Separate Channels of Image 1}
\label{fig:tuning-image1}
\end{figure}
\begin{table}[h!]
\centering
\caption{Comparison of SNRs - Image 1 (reference : fig. \ref{fig:zebrafish_0})}
\begin{tabular}{*5c}
\toprule
Channel &  \multicolumn{2}{c}{Before Deconvolution} & \multicolumn{2}{c}{After Deconvolution}\\
\midrule
{}   & SNR   & Reliability    & SNR   & Reliability \\
0   &  15 & Very good   & 16  & Very Good\\
1   &  25 & Mediocre & 16  & Very Good\\
2   &  16  &  Mediocre & 16  & Very Good\\
\bottomrule
\end{tabular}
\end{table}
\subsubsection*{Image 2}
Apart from the parameters mentioned above, the values of the signal garbage volume was changed to 100493. 
\begin{figure}[h!]
\centering
\subfloat{\label{fig:Before Tuning 2}\includegraphics[width=0.45\textwidth]{Report/Images/original/Image_for_Huygens_1_rendering.jpg}}
\vspace{5 mm}
\subfloat{\label{fig:After Tuning 2}\includegraphics[width=0.45\textwidth]{Report/Images/cleaned-up/Image_for_Huygens_1_rendering.jpg}}
\caption{Processing Separate Chanels of Image 2}
\label{fig:tuning-image2}
\end{figure}
\begin{table}[h!]
\centering
\caption{Comparison of SNRs - Image 2 (reference : fig. \ref{fig:zebrafish_1})}
\begin{tabular}{*5c}
\toprule
Channel &  \multicolumn{2}{c}{Before Deconvolution} & \multicolumn{2}{c}{After Deconvolution}\\
\midrule
{}   & SNR   & Reliability    & SNR   & Reliability \\
0   &  16 & Very Good & 16 & Very Good \\
1   &  16 & Very Good & 16 & Very Good \\
2   &  16 & Very Good & 15 & Very Good \\
\bottomrule
\end{tabular}
\end{table}
\subsubsection*{Image 3}
Apart from the parameters mentioned above, the values of the signal garbage volume was changed to 770303. 
\begin{figure}[h!]
\centering
\subfloat{\label{fig:Before Tuning 3}\includegraphics[width=0.45\textwidth]{Report/Images/original/Image_for_Huygens_2_rendering.jpg}}
\vspace{5 mm}
\subfloat{\label{fig:After Tuning 3}\includegraphics[width=0.45\textwidth]{Report/Images/cleaned-up/Image_for_Huygens_2_rendering.jpg}}
\caption{Processing Separate Channels of Image 3}
\label{fig:tuning-image3}
\end{figure}
\begin{table}[h!]
\centering
\caption{Comparison of SNRs - Image 3 (reference : fig. \ref{fig:zebrafish_2})}
\begin{tabular}{*5c}
\toprule
Channel &  \multicolumn{2}{c}{Before Deconvolution} & \multicolumn{2}{c}{After Deconvolution}\\
\midrule
{}   & SNR   & Reliability    & SNR   & Reliability \\
0   &  16 & Very Good & 16 & Very Good \\
1   &  16 & Very Good & 16 & Very Good \\
2   &  16 & Very Good & 16 & Very Good \\
\bottomrule
\end{tabular}
\end{table}
\subsubsection*{Image 4}
Apart from the parameters mentioned above, the values of the signal garbage volume was changed to 11600.
\begin{figure}[h!]
\centering
\subfloat{\label{fig:Before Tuning 4}\includegraphics[width=0.45\textwidth]{Report/Images/original/Image_for_Huygens_3_rendering.jpg}}
\vspace{5 mm}
\subfloat{\label{fig:After Tuning 4}\includegraphics[width=0.45\textwidth]{Report/Images/cleaned-up/Image_for_Huygens_3_rendering.jpg}}
\caption{Processing Separate Channels of Image 4}
\label{fig:tuning-image4}
\end{figure}
\begin{table}[h!]
\centering
\caption{Comparison of SNRs - Image 4 (reference : fig. \ref{fig:zebrafish_tail})}
\begin{tabular}{*5c}
\toprule
Channel &  \multicolumn{2}{c}{Before Deconvolution} & \multicolumn{2}{c}{After Deconvolution}\\
\midrule
{}   & SNR   & Reliability    & SNR   & Reliability \\
0   &  16 & Very Good & 11 & Very Good \\
1   &  16 & Very Good & 16 & Very Good \\
2   &  16 & Very Good & 15 & Very Good \\
\bottomrule
\end{tabular}
\end{table}

\clearpage
\subsection*{Automated Deconvolution}
\subsubsection*{Image 1}
Image 1 was automatically deconvolved and the resultant image showed a significant improvement in SNR. Furthermore the reliability of the channels improved from fair-mediocre to very good.
\begin{figure}[h!]
\centering
\subfloat{\label{fig:Before Deconvolution 1}\includegraphics[width=0.45\textwidth]{Report/Images/original/Image_for_Huygens_0_rendering.jpg}}
\vspace{5 mm}
\subfloat{\label{fig:After Deconvolution 1}\includegraphics[width=0.45\textwidth]{Report/Images/auto-deconvolved/Image_for_Huygens_0_decon_rendering.jpg}}
\caption{Automatic deconvolution of Image 1}
\label{fig:auto-deconvolve-image1}
\end{figure}
\begin{table}[h!]
\centering
\caption{Comparison of SNRs - Image 1 (reference : fig. \ref{fig:zebrafish_0})}
\begin{tabular}{*5c}
\toprule
Channel &  \multicolumn{2}{c}{Before Deconvolution} & \multicolumn{2}{c}{After Deconvolution}\\
\midrule
{}   & SNR   & Reliability    & SNR   & Reliability \\
0   &  15 & Very good   & 15.96  & Good\\
1   &  25 & Mediocre & 15.74  & Fair\\
2   &  16  &  Mediocre & 19.98  & Mediocre\\
\bottomrule
\end{tabular}
\end{table}
\subsubsection*{Image 2}
Similarly, image 2 was automatically deconvolved and the SNRs were noted. Though the image did not show any improvements in the SNR, further examination revealed that the intensities dropped slightly. 
\begin{figure}[h!]
\centering
\subfloat{\label{fig:Before Deconvolution 2}\includegraphics[width=0.45\textwidth]{Report/Images/original/Image_for_Huygens_1_rendering.jpg}}
\vspace{5 mm}
\subfloat{\label{fig:After Deconvolution 2}\includegraphics[width=0.45\textwidth]{Report/Images/auto-deconvolved/Image_for_Huygens_1_decon_rendering.jpg}}
\caption{Automatic deconvolution of Image 2}
\label{fig:auto-deconvolve-image2}
\end{figure}
\begin{table}[h!]
\centering
\caption{Comparison of SNRs - Image 2 (reference : fig. \ref{fig:zebrafish_1})}
\begin{tabular}{*5c}
\toprule
Channel &  \multicolumn{2}{c}{Before Deconvolution} & \multicolumn{2}{c}{After Deconvolution}\\
\midrule
{}   & SNR   & Reliability    & SNR   & Reliability \\
0   &  16 & Very Good & 16 & Very Good \\
1   &  16 & Very Good & 16 & Very Good \\
2   &  16 & Very Good & 16 & Very Good \\
\bottomrule
\end{tabular}
\end{table}
\subsubsection*{Image 3}
Image 3 was automatically deconvolved and the SNRs were noted. Though the image did not show any improvements in the SNR (as the previous one), further examination revealed that the intensities dropped slightly. 
\begin{figure}[h!]
\centering
\subfloat{\label{fig:Before Deconvolution 3}\includegraphics[width=0.45\textwidth]{Report/Images/original/Image_for_Huygens_2_rendering.jpg}}
\vspace{5 mm}
\subfloat{\label{fig:After Deconvolution 3}\includegraphics[width=0.45\textwidth]{Report/Images/auto-deconvolved/Image_for_Huygens_2_decon_rendering.jpg}}
\caption{Automatic deconvolution of Image 3}
\label{fig:auto-deconvolve-image3}
\end{figure}
\begin{table}[h!]
\centering
\caption{Comparison of SNRs - Image 3 (reference : fig. \ref{fig:zebrafish_2})}
\begin{tabular}{*5c}
\toprule
Channel &  \multicolumn{2}{c}{Before Deconvolution} & \multicolumn{2}{c}{After Deconvolution}\\
\midrule
{}   & SNR   & Reliability    & SNR   & Reliability \\
0   &  16 & Very Good & 16 & Very Good \\
1   &  16 & Very Good & 16 & Very Good \\
2   &  16 & Very Good & 16 & Very Good \\
\bottomrule
\end{tabular}
\end{table}
\subsubsection*{Image 4}
Image 4 (the zebrafish tail) was automatically deconvolved and the SNRs were noted. Though the image did not show any improvements in the SNR (as the previous ones), further examination revealed that the intensities dropped slightly. 
\begin{figure}[h!]
\centering
\subfloat{\label{fig:Before Deconvolution 4}\includegraphics[width=0.45\textwidth]{Report/Images/original/Image_for_Huygens_3_rendering.jpg}}
\vspace{5 mm}
\subfloat{\label{fig:After Deconvolution 4}\includegraphics[width=0.45\textwidth]{Report/Images/auto-deconvolved/Image_for_Huygens_3_decon_rendering.jpg}}
\caption{Automatic deconvolution of Image 4}
\label{fig:auto-deconvolve-image4}
\end{figure}
\begin{table}[h!]
\centering
\caption{Comparison of SNRs - Image 4 (reference : fig. \ref{fig:zebrafish_tail})}
\begin{tabular}{*5c}
\toprule
Channel &  \multicolumn{2}{c}{Before Deconvolution} & \multicolumn{2}{c}{After Deconvolution}\\
\midrule
{}   & SNR   & Reliability    & SNR   & Reliability \\
0   &  16 & Very Good & 16 & Very Good \\
1   &  16 & Very Good & 16 & Very Good \\
2   &  16 & Very Good & 16 & Very Good \\
\bottomrule
\end{tabular}
\end{table}
\clearpage
\subsection*{Manual Deconvolution}
In the following section we applied manual deconvolution to all 4 images of the Zebrafish. Manual deconvolution has more parameters that need to be manually fine-tuned such as the deconvolution strategy, the background threshold, the number of iterations, and the deconvolution layout. 
\subsubsection*{Image 1}
An aggressive strategy for deconvolution was used here with Classic MLE algorithm. An automatic estimation of the background with lowest background value as 0.1, as there is not a lot of background space. This was done over 50 iterations and the acuity of the image was increased to +90, to prevent dataloss. \newline
\begin{figure}[h!]
\centering
\subfloat{\label{fig:Before Manual Deconvolution 1}\includegraphics[width=0.45\textwidth]{Report/Images/original/Image_for_Huygens_0_rendering.jpg}}
\vspace{5 mm}
\subfloat{\label{fig:After Manual Deconvolution 1}\includegraphics[width=0.45\textwidth]{Report/Images/manual-deconvolved/manual_deconvolution_image0_rendering.jpg}}
\caption{Manual deconvolution of Image 1}
\label{fig:manual-deconvolve-image1}
\end{figure}
\\ 
Channel 1: 3175 clipped voxels detected, indicating significant data loss or over-processing. \\
Channel 2: 149 clipped voxels detected, indicating moderate data loss or over-processing.\\
The data loss is seen in the histograms. However, the intensity of the image seems to be reduced, especially in channel 1, as the intensities are lower in the manually deconvolved image. 
\begin{figure}[h!]
\centering
\subfloat{\label{fig:Before Manual Deconvolution Histogram 1}\includegraphics[width=0.45\textwidth]{Report/Images/manual-deconvolved/original_image1_graph.png}}
\vspace{5 mm}
\subfloat{\label{fig:After Manual Deconvolution Histogram 1}\includegraphics[width=0.45\textwidth]{Report/Images/manual-deconvolved/deconvolved_image1_graph.png}}
\caption{Histograms of Image 1 - Before and After Manual Deconvolution}
\label{fig:manual-deconvolve-image1-histogram}
\end{figure}
\subsubsection*{Image 2}
A similar aggressive strategy was used for all the channels, but with Good's MLE. An automatic estimation of the background with lowest value as 0.1 was used, as there is not a lot of space. As Good's MLE was used, only 20 iterations were done, with the sharpness increased to +90, the quality threshold set at 0.3 and the brick layout for deconvolution at 1. 
\begin{figure}[h!]
\centering
\subfloat{\label{fig:Before Manual Deconvolution 2}\includegraphics[width=0.45\textwidth]{Report/Images/original/Image_for_Huygens_1_rendering.jpg}}
\vspace{5 mm}
\subfloat{\label{fig:After Manual Deconvolution 2}\includegraphics[width=0.45\textwidth]{Report/Images/manual-deconvolved/manual_deconvolution_image1_rendering.jpg}}
\caption{Manual deconvolution of Image 2}
\label{fig:manual-deconvolve-image2}
\end{figure}
\\ 
Channel 0: 22 clipped voxels detected, less data loss than Image 0.\\ Other channels did not show any data loss. 
As there is no significant data loss, the histograms seems to be similar, with a miniscule change in intensities in channel 2. 
\begin{figure}[h!]
\centering
\subfloat{\label{fig:Before Manual Deconvolution Histogram 2}\includegraphics[width=0.45\textwidth]{Report/Images/manual-deconvolved/original_image2_graph.png}}
\vspace{5 mm}
\subfloat{\label{fig:After Manual Deconvolution Histogram 2}\includegraphics[width=0.45\textwidth]{Report/Images/manual-deconvolved/deconvolved_image2_graph.png}}
\caption{Histograms of Image 2 - Before and After Manual Deconvolution}
\label{fig:manual-deconvolve-image2-histogram}
\end{figure}\newline
\subsubsection*{Image 3} 
Similar to Image 1, Classic MLE algorithm was used, with a conservative strategy for deconvolution. 
Channels 0 and 1 use a conservation strategy with Classic MLE, automatic estimation of background was used, with in/near object background value set to 0.05. A classic strategy of iteration with 72 iterations was used, with the brick layout set to slice-by-slice layout. The sharpness and quality threshold remained unchanged \newline
\begin{figure}[h!]
\centering
\subfloat{\label{fig:Before Manual Deconvolution 3}\includegraphics[width=0.45\textwidth]{Report/Images/original/Image_for_Huygens_2_rendering.jpg}}
\vspace{5 mm}
\subfloat{\label{fig:After Manual Deconvolution 3}\includegraphics[width=0.45\textwidth]{Report/Images/manual-deconvolved/manual_deconvolution_image2_rendering.jpg}}
\caption{Manual deconvolution of Image 3}
\label{fig:manual-deconvolve-image3}
\end{figure}
\\ 
Channel 0: 141 clipped voxels detected.\\ Other channels did not show any data loss. The histograms however, show drop in certain intensities in channel 0 and channel 1. This is noticeable in the image itself, with noise missing in the deconvolved image (in the eye). 
\begin{figure}[h!]
\centering
\subfloat{\label{fig:Before Manual Deconvolution Histogram 3}\includegraphics[width=0.45\textwidth]{Report/Images/manual-deconvolved/original_image3_graph.png}}
\vspace{5 mm}
\subfloat{\label{fig:After Manual Deconvolution Histogram 3}\includegraphics[width=0.45\textwidth]{Report/Images/manual-deconvolved/deconvolved_image3_graph.png}}
\caption{Histograms of Image 3 - Before and After Manual Deconvolution}
\label{fig:manual-deconvolve-image3-histogram}
\end{figure}\newline
\subsubsection*{Image 4}
As the image of the zebrafish tail, similar parameters as seen in Image 2 are used. However, minor changes were made, such as the background estimation set to 0.5, as there is a lot of background in this image. Furthermore, the quality threshold was set to 0.5, to ensure a good output, and the brick layout to many bricks.  \newline
\begin{figure}[h!]
\centering
\subfloat{\label{fig:Before Manual Deconvolution 4}\includegraphics[width=0.45\textwidth]{Report/Images/original/Image_for_Huygens_3_rendering.jpg}}
\vspace{5 mm}
\subfloat{\label{fig:After Manual Deconvolution 4}\includegraphics[width=0.45\textwidth]{Report/Images/manual-deconvolved/manual_deconvolution_image3_rendering.jpg}}
\caption{Manual deconvolution of Image 4}
\label{fig:manual-deconvolve-image4}
\end{figure}
\\ 
Channel 2: 14014 clipped voxels detected, indicating severe data loss or over-processing, possibly due to the high number of iterations and aggressive strategy.\\ Only channel 2 exhibited a significant data loss. Other channels did not show any data loss. \\The histograms however, show drop in certain intensities in channel 0. 
\begin{figure}[h!]
\centering
\subfloat{\label{fig:Before Manual Deconvolution Histogram 4}\includegraphics[width=0.45\textwidth]{Report/Images/manual-deconvolved/original_image4_graph.png}}
\vspace{5 mm}
\subfloat{\label{fig:After Manual Deconvolution Histogram 4}\includegraphics[width=0.45\textwidth]{Report/Images/manual-deconvolved/deconvolved_image4_graph.png}}
\caption{Histograms of Image 4 - Before and After Manual Deconvolution}
\label{fig:manual-deconvolve-image4-histogram}
\end{figure}
\\
The different settings impact the image quality and data preservation. Aggressive strategies and higher iterations often lead to more clipped voxels, indicating potential over-processing and loss of detail. Conservation strategies with fewer iterations tend to preserve more data but may result in less aggressive noise reduction and sharpening. In the end it is a trade-off between more conservation of data (less clipped voxels) and higher quality (sharpening and noise reduction). The choice of deconvolution algorithm (Classic MLE vs. Goode MLE) also influences the outcome, with Goode MLE typically requiring fewer iterations for similar results.

\subsection*{Visualizations}
The SFP visualizations of all 4 images were generated using Huygens Movie Maker. 
\\The parameters were changed as detailed in Question 4.1. 
\medskip\\
\textbf{SFP of Image 1} : File Name - \textit{A603\_Image3\_SFP.mp4}\\
\textbf{SFP of Image 2} : File Name - \textit{A603\_Image4\_SFP.mp4}\\
\textbf{SFP of Image 3} : File Name - \textit{A603\_Image1\_SFP.mp4}\\
\textbf{SFP of Image 4} : File Name - \textit{A603\_Image2\_SFP.mp4}\\

\section*{Appendix}
\subsection*{Function to display image content in planes}
\begin{figure}[h!]
    \centering
    \includegraphics[width=1\linewidth]{Report/Images/3d_plot.png}
    \caption{The results of the function to display slices of tif image}
    \label{fig:3d-plane-image}
\end{figure}
% \bibliographystyle{alpha}
% \bibliography{sample}

\end{document}